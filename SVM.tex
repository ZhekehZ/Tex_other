\documentclass{article}
\usepackage[T2A]{fontenc}
\usepackage[utf8]{inputenc}
\usepackage[english,russian]{babel}
\usepackage{amsmath}
\usepackage{amsfonts}
\usepackage{amssymb}
\usepackage{makeidx}

\begin{document}
\section*{SVM}
Пусть $X = \mathbb{R}$ - пространство объектов, $Y = \{-1, 1\}$ - классы, $X^l=\{(x_i, y_i)\}_{i=1}^l$. Требуется построить классификатор $a: X \rightarrow Y$.\\

Линейный пороговый классификатор:
$$a(x) = sign(\langle w, x\rangle - w_0)$$
Уравнение $\langle w, x\rangle = w_0$ описывает гиперплоскость, разделяющую классы.\\
Функционал ошибок:
$$Q(w, w_0) = \sum_{i=1}^n\Big( y_i(\langle w, x_i\rangle - w_0) < 0\Big)$$

Нормировка:\\
Пусть $x_k$ - наиболее близкие к гиперплоскости из набора $\{x_i\}$. Тогда возьмем такие $w, w_0$, что $\langle w, x_k\rangle - w_0 = y_k$. Тогда для всех $x_i$:
\[ \langle w, x_k\rangle - w_0 \left\{
\begin{array}{ll}
\leq -1, & y_i = -1\\
\geq 1, &  y_i = 1
\end{array} \right. \Leftrightarrow m_i = y_i(\langle w, x_k\rangle - w_0) \geq 1\]
Необходимо найти такие параметры, при которых ширина разделяющей полосы максимальна. Пусть $x_1$ - ближайший к гиперплоскости элемент класса $\{1\}$, $x_{-1}$ - класса $\{-1\}$. Тогда ширина полосы: 
$$\langle (x_1 - x_{-1}), \frac{w}{||w||}\rangle = \frac{(w_0 + 1) - (w_0 - 1)}{||w||} = \frac{2}{||w||} \rightarrow sup$$

Необходимо решить следующую экстремальную задачу:\\
\[\left\{
\begin{array}{ll}
||w|| \rightarrow inf\\
y_i(\langle w, x_i\rangle - w_0) \geq 1, & i\in 1:l
\end{array} \right.\]
В результате получается, что $w$ может быть представлена как линейная комбинация $x_i$, лежащих на границе разделяющей полосы. 

Если выборка линейно неразделима:\\
\[\left\{
\begin{array}{ll}
\frac{1}{2}||w|| + C \sum_{i=1}^l\varepsilon_i \rightarrow inf\\
y_i(\langle w, x_i\rangle - w_0) \geq 1 - \varepsilon_i, & i\in 1:l\\
\varepsilon_i \geq 0
\end{array} \right.\]
$$Q = \sum_{i=1}^l\Big[ m_i < 0\Big]$$
Замеим пороговую функцию на ее верхней оценкой $\Big[ m_i < 0\Big] \leq (1 - m_i)_+$ и добавим слагаемое регуляризации:
$$Q = \sum_{i=1}^l(1 - m_i)_+ + \tau ||w||^2 \rightarrow inf$$
В результате снова получается, что $w$ можнт быть представлена в виде линейной комбинации векторов на границе разделяющей полосы.
\[\left\{
\begin{array}{ll}
-\sum_{i=1}^l \lambda_i + \frac{1}{2}\sum_{i=1}^l\sum_{j=1}^l \lambda_i\lambda_j\langle x_i, x_j\rangle \rightarrow inf\\
0 \leq \lambda_i \leq C\\
\sum_{i=1}^l \lambda_iy_i = 0
\end{array} \right.\]
	
Ядро\\Ядром называется функция $K(x, x')$, если она симметрична и ноетрицательно определена.\\ 	
	
Во всех приведенных выражениях скалярное произведение можно заменить ядром $K$.
\end{document}